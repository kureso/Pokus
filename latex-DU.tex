\documentclass[10pt,a4paper]{article}
\usepackage{amsmath}
\usepackage{amssymb}
\usepackage{amsthm}
\usepackage{amsfonts}

% PAGE DIMENSION
%\usepackage[left=2cm,right=2cm,top=2cm,bottom=2cm]{geometry}

\usepackage[margin=2cm,headsep=0.8cm,headheight=13pt]{geometry}

% ENCODING, LANGUAGE
\usepackage[czech]{babel}
\usepackage[utf8]{inputenc}

% GRAPHICS
\usepackage{graphicx,graphics}

%Head/Foot Fancy
\usepackage{fancyhdr}
\usepackage{lastpage}

\pagestyle{fancy}
\fancyhead[L]{\textbf{DÚ 2}}
\fancyhead[R]{NOFY162 Matematika pro fyziky 2, \textbf{Ladislav Trnka}}
\fancyfoot[C]{\bfseries \thepage/\pageref{LastPage}}
\renewcommand{\headrulewidth}{0.4pt}

%JEn tak
\usepackage[T1]{fontenc}
\usepackage{lmodern}

% HYPERTEXT, SOURCE CODE SPECIALS
\usepackage[unicode]{hyperref}
\usepackage[active]{srcltx} % use TeX-souce-specials-mode


% UNITS, TYPESETTING TENSORS
\usepackage{units}
\usepackage{tensor}
\usepackage{accents}

% COMPACT LIST ENVIRONMENT
\usepackage{enumitem}

% LINE NUMBERS
\usepackage{lineno}

% TABLE OF CONTENTS IN TWO COLUMNS
% \usepackage[toc]{multitoc} % It seems that it does not work with amsart
% the workaround is the command
% \addtocontents{toc}{\protect\begin{multicols}{2}} % workaround for table of contents in two columns in amsart documentclass
% see below
\usepackage{multicol}

% SELECTIVELY INCLUDE/EXCLUDE PARTS OF TEXT
\usepackage{comment}

% FLOAT BARRIER
\usepackage{placeins}

\input{macros-experimental}

\begin{document}


Máme funkci $u(x,y)$:
\begin{equation}
\boxed{
u(x,y) = x^2-y^2+5x+y-\frac{y}{x^2+y^2}.
}
\end{equation}
Protože $(x,y) \neq (0,0)$, funkce $u(x,y)$ je definována na oblasti $\mathbb{R}^{2}\setminus\lbrace 0\rbrace$, navíc je $u(x,y) \in \mathrm{C}^{2}(\mathbb{R}^{2}\setminus\lbrace 0\rbrace)$.
\begin{description}
\item[(1)] Zadaná funkce $u(x,y)$ je harmonická na $\mathbb{R}^{2}\setminus\lbrace 0\rbrace$:
\begin{equation}
\begin{split}
\Delta u &= \ppd u x + \ppd u y  =  \frac{\partial}{\partial x} \left( 2 x+5 +\frac{2 x y}{\left(x^{2}+y^{2}\right)^{2}} \right) +
\frac{\partial}{\partial y} \left( -2y+1-\frac{x^{2}-y^{2}}{\left(x^{2}+y^{2} \right)^{2}}\right) =\\
 & = \dfrac{4y(x^{2}+y^{2})-8yx^{2}+4y(x^{2}-y^{2})}{(x^{2}+y^{2})^{3}} =0. \\
\end{split}
\end{equation}
Harmonicky sdružená funkce $v(x,y)$ splňuje Cauchy-Riemannovy podmínky:
\begin{equation}
\label{CR}
\begin{split}
\pd v y & = \pd u x =  2 x+5 +\frac{2 x y}{\left(x^{2}+y^{2}\right)^{2}} ,\\
\pd v x & = -\pd u y = 2y-1+\frac{x^{2}-y^{2}}{\left(x^{2}+y^{2} \right)^{2}} .\\
\end{split}
\end{equation}
Integrací první rovnice v $\eqref{CR}$ získáváme:
\begin{equation}
v(x,y)= \int 2 x+5 +\frac{2 x y}{\left(x^{2}+y^{2}\right)^{2}} \diff y = 2xy+5y-\dfrac{x}{x^{2}+y^{2}}+C(x),
\end{equation}
kde $C(x)$ je funkce. Dosazením do druhé dostáváme:
\begin{equation}
\begin{split}
\pd v x & = -\pd u y  ,\\
2y -\dfrac{y^{2}-x^{2}}{(x^{2}+y^{2})^{2}} +C'(x)& = 2y-1+\frac{x^{2}-y^{2}}{\left(x^{2}+y^{2} \right)^{2}} ,\\
C'(x) & = -1 ,\\
C(x) & = -x+C .\\
\end{split}
\end{equation}
Harmonicky sdružená funkce k $v(x,y)$ je:
\begin{equation}
\boxed{
v(x,y) = 2xy+5y-\dfrac{x}{x^{2}+y^{2}}-x+C .
}
\end{equation}
\item[(2)] Nalezená funkce $v(x,y)$ je skutečně harmonická na $\mathbb{R}^{2}\setminus\lbrace 0\rbrace$:
\begin{equation}
\begin{split}
\Delta v &= \ppd v x + \ppd v y  =  \frac{\partial}{\partial x}
\left( 2y - \frac{y^{2}-x^{2}}{(x^{2}+y^{2})^{2}} -1 \right)+\frac{\partial}{\partial y} \left(2x+5 +\frac{2xy}{(x^{2}+y^{2})^{2}} \right) =  \\
& = \dfrac{4x(x^{2}+y^{2})-4x(x^{2}-y^{2})-8xy^{2}}{(x^{2}+y^{2})^{3}}=0 .\\
\end{split}
\end{equation}
\item[(3)] Explicitní tvar funkce $f(z)$ definované z $u$ a $v$:
\begin{equation}
\begin{split}
f(z) & =u(\operatorname{Re} z,\operatorname{Im} z)+ v(\operatorname{Re} z,\operatorname{Im} z) = u(x,y)+v(x,y) ,\\
f(z) & = x^2-y^2+5x+y-\frac{y}{x^2+y^2} +i (2xy+5y-\dfrac{x}{x^{2}+y^{2}}-x+C) ,\\
f(z) & = (x^{2}-y^{2}+2xyi)+5(x+iy)-\frac{y+ix}{x^{2}+y^{2}}+y-ix+iC ,\\
f(z) & = z^{2}+5z-\frac{i\overline{z}}{|z|}-iz+iC ,\\
\end{split} 
\end{equation}
\begin{equation}
\boxed{f(z) = z^{2}+(5-i)z-\frac{i}{z}+iC.}
\end{equation}
Funkce $f(z)$ je holomorfní na $\mathbb{C}\setminus\lbrace 0\rbrace$, protože je součtem polynomu (holomorfní na $\mathbb{C}$) a racionální funkce (až na kořeny polynomu ve jmenovateli holomorfní funkce).
\end{description}
\end{document}
