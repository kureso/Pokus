\documentclass{article}
\usepackage{amsmath}
\usepackage{amssymb}
\begin{document}
Nejprve začneme s hledáním dalšího členu rozvoje pro periodu kmitů. NOVINKA NA TRHU. ZKOUŠKA!!!!
Zatím jsme našli:
\begin{equation*}
T \approx 2\pi \sqrt{\frac{l}{g}}\left(1+\frac{\theta_{init}^2}{16}\right)
\end{equation*}
Koukneme jak vypadá rozvoj $\frac{1}{\sqrt{1-x}}$ až do třetího člene:
\begin{equation*}
\frac{1}{\sqrt{1-x}} = 1+\frac{x}{2}+\frac{3 x^2}{8}+O\left(x^3\right)
\end{equation*}
Když za x dosadíme $k^2 \sin ^2(\phi )$ dostaneme (zanedbáváme zbytek):
\begin{equation}
1 + \frac{1}{2}k^2\sin^2(\phi)+\frac{3}{8}k^4\sin^4(\phi) \tag{*}
\end{equation}
Máme tu takovou malou důležitou rovnost:
\begin{equation*}
k = \sin\frac{\theta_{init}}{2}
\end{equation*}
Pro naše k bychom mohli udělat rozvoj, ale jen k nebude stačit, neboť ve výrazu (*) máme vyšší mocniny k, takže uděláme rovoj pro vyšší mocniny k.
\begin{align*}
k^2 = \sin^2\frac{\theta_{init}}{2} \approx \frac{\theta ^2}{4}+O\left(\theta ^3\right)\\
k^4 =  \sin^4\frac{\theta_{init}}{2} \approx \frac{\theta ^4}{16}+O\left(\theta ^5\right)
\end{align*}
Udělali jsme rozvoj do prvního členu, ty další nás nezajímají :D
Teď to naházíme do jednoho kastrolu a zamícháme. Získáme toto:
\begin{equation*}
\frac{1}{\sqrt{1-k^2\sin^2\phi}} \approx 1+\frac{1}{8} \theta ^2 \sin ^2(\phi )+\frac{3}{128} \theta ^4 \sin ^4(\phi )
\end{equation*}
Teď už zapíšeme periodu T s novým členem.
\begin{align*}
T &= 4\sqrt{\frac{l}{g}}\int_{0}^{\frac{\pi}{2}}\frac{\,d\phi}{\sqrt{1-k^2\sin^2\phi}}\approx 4\sqrt{\frac{l}{g}}\int_{0}^{\frac{\pi}{2}}\left(1+\frac{1}{8} \theta ^2 \sin ^2(\phi )+\frac{3}{128} \theta ^4 \sin ^4(\phi )\right)\,d\phi= \\
&= \frac{1}{512} \pi  \sqrt{\frac{l}{g}} \left(1024+64 \theta ^2+9 \theta ^4\right)
\end{align*}
\end{document}
